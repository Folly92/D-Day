\documentclass[9pt]{beamer}
\usetheme{CambridgeUS}
\usepackage{textpos}
\usepackage[latin1]{inputenc}
\usepackage{amsmath}
\usepackage{mathtools}
\usepackage{tikz}
\usetikzlibrary{arrows,shapes}
\usecolortheme{beaver}
\usepackage{graphicx}
\usepackage{comment} 

\title{ Github intro }
\author{\Large Francesco Maria Follega \\{\small francesco.follega@unitn.it}}
\date{\today}
\institute[]{\includegraphics[width=8cm,height=2.2cm]{logo_TU}}
\setbeamertemplate{navigation symbols}{}
\setbeamertemplate{navigation symbols}{}
%\setbeamertemplate{footline}{}


\begin{document}

\begin{frame}
	\maketitle
\end{frame}

\begin{frame}{Very first steps - Set up a GitHub account}
In the following slide, there is a small list of steps to compete in order to start working on the \textbf{GitHub} ( improved web integrated version of the previous \text{Git} ).
\begin{itemize}
\item subscribe here \textit{https://github.com};
\item once you have created an account send me an email with \textbf{your account nickname}, so that I can invite you as a collaborator;
\item join as a developer of repository \textit{"Limadou"}.
\end{itemize}
Now you are a collaborator of the "Limadou" project and you can produce new versions of the data analysis Package and make it available for usage or further developments by the other collaborators. \newline \newline
\textbf{NB: the all code is completely \textit{open source}, so it can be read by other GitHub user, but it is editable just by the collaborators.}

\end{frame}

\begin{frame}{Let's start! - Set up a local directory}
\item 
\end{frame}

\end{document}




